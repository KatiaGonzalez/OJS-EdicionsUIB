
\documentclass[9pt]{IEEEtran}
%\usepackage[spanish]{babel}	

\usepackage{latexsym,array,epsfig}
\usepackage{amsmath}
\usepackage{amsfonts}
\usepackage{amssymb}
\usepackage{makeidx}
\usepackage{algorithm}
\usepackage{longtable}

\usepackage{theorem}
\usepackage{graphicx}
\usepackage{graphics}
\setlength{\topmargin}{-2cm}
\setlength{\textwidth}{16cm}
\setlength{\textheight}{24cm}
\setlength{\oddsidemargin}{0cm}
\usepackage[utf8]{inputenc}
\usepackage{booktabs,enumerate,graphicx,epstopdf}
\usepackage{amsfonts,amssymb,amsmath,amsthm,wasysym}
\usepackage[calcwidth]{titlesec}
\usepackage{hyperref} 
  \usepackage[catalan]{babel}
 \selectlanguage{catalan}
  \usepackage{color}
\parindent = 0 pt


\begin{document}

\pagestyle{empty}
%
\title{Una estratègia òptima per al joc del 7 i mig}
\author{Biel Llinàs Sansó\\
\vspace{0.5cm}
\textit{Treball acadèmicament dirigit \\ optativa de la Llicenciatura de Matemàtiques}}
\maketitle
\thispagestyle{empty}

\begin{abstract}
En aquest article s'estudia un joc de cartes molt popular en els bars de Mallorca, el 7 i mig. El 7 i mig és un joc semblant al Black-Jack que es juga amb la baralla espanyola, l'objectiu és demanar cartes intentant acostar-se el màxim possible a 7 punts i mig, però sense passar-se.

Tractarem aquest joc intentant maximitzar el guany esperat i, mitjançant un procés d'inducció per enrere i sota certes hipòtesis, obtendrem una regla d'aturada òptima.
\end{abstract}


\section{Regles del joc}

El joc es juga amb la baralla espanyola sense els vuits i els nous, cada carta entre 1 i 7 té el valor del número que té associat, les demés cartes (10, 11 i 12) valen mig punt, l'objectiu serà acostar-se el màxim possible, però sense passar-se, a 7 punts i mig.

Abans de repartir les cartes, cada jugador ha de fer una aposta mínima, anomenada cega, per entrar en el joc. En el present article suposarem que aquesta cega és de 1 euro. El jugador que reparteix les cartes és la banca, aquesta reparteix una carta tapada a cada jugador i una per a ella. Una vegada vista la carta, els jugadors han de decidir si volen més cartes o si volen plantar-se. En el cas que un jugador decideixi demanar una carta més, tendrà dret a mantenir una carta tapada oculta per a la banca i a augmentar la seva aposta (abans de veure la nova carta). En el cas que un jugador es passi de 7 punts i mig, ha d'ensenyar les seves cartes i perd automàticament l'aposta realitzada. Finalment, si les cartes sumen 7 punts, el jugador pot elegir fer un enterrament, això és, destapar totes les seves cartes i demanar una carta tapada, que no es destaparà fins que la banca no hagi efectuat el seu joc.

Després del joc dels jugadors, efectua el seu joc la banca i, posteriorment, es realitza la ronda de pagaments. La banca guanya l'aposta realitzada per cada jugador a qui supera o iguala els punts; en cas contrari, la banca paga l'aposta realitzada al jugador corresponent. Si un jugador ha aconseguit 7 punts i mig, la banca paga el doble de l'aposta realitzada.

\section{Metodologia}

Començarem l'estudi del joc sota les següents hipòtesis:
\begin{itemize}
\item $H_1$: La probabilitat d'obtenir una carta concreta no varia al llarg del joc, és a dir, la probabilitat d'obtenir un mig és de $3/10$ i la d'obtenir qualsevol altre carta és $1/10$
\item $H_2$: El total de punts de la banca és independent del que fan els diferents jugadors.
\end{itemize}
La hipòtesi $H_1$ és una simplificació necessària en una primera etapa d'anàlisi del joc; aquesta és raonable si es té en compte que la banca utilitza una baralla de cartes suficientment gran. Amb la hipòtesi $H_2$ assumim que la banca fixa una estratègia i no la veu alterada per les cartes i les apostes dels jugadors.

Sota aquestes hipòtesis,  suposarem que, en un moment donat del joc el jugador coneix la suma de les seves cartes, $x$, i l'estratègia que seguirà la banca, $b$ (puntuació a partir de la qual s'aturarà). Davant una situació donada $(x,b)$, el jugador ha de decidir si plantar-se o demanar una altra carta.

Utilitzarem la notació següent:
\begin{itemize}
\item $G^*(x,b)$ és el màxim guany esperat pel jugador en la situació $(x,b)$ que s'obté quan s'actua de forma òptima.
\item $G_0(x,b)$ és el guany esperat si a la vista de $(x,b)$ el jugador decideix plantar-se.
\item $P_c$ és la probabilitat d'obtenir una carta de valor $c$; segons $H_1$ aquesta probabilitat es mantendrà constant al llarg de tota la partida i val
$$P_c = \begin{cases}
3/10 & \text{si } c = 1/2 \\
1/10 & \text{si } c \not = 1/2
\end{cases}$$
\end{itemize}
Per tant, si anomenam 
$$G_1^*(x,b) =  P_{1/2}\cdot G^*(x + 1/2,b) + \sum_{c=1}^7P_c \cdot G^*(x+c,b)$$
podem expressar 
\begin{equation}
G^*(x,b) = \max \left\{G_0(x,b) , G_1^*(x,b)\right\}
\label{eqGEstrella}
\end{equation}

És a dir, comparam el guany esperat si ens aturam i el guany esperat si es demana una carta i després es continua el joc seguint l'estratègia òptima. Convendrà aturar-se, doncs, quan el màxim s'agafi en $G_0(x,b)$ i demanar una carta en cas contrari.

Hem d'avaluar l'equació anterior per a cada $x$ i per a cada $b$. Per això necessitam conèixer els valors de $G_0(x,b)$ per a cada $x$ i per a cada $b$ i els valors finals de $G^*(x,b)$ necessaris per a poder iniciar el procés d'inducció cap enrere amb qualsevol valor de $x$ i $b$.

\section{Càlcul de l'estratègia òptima}

Tenim que els valors de $G_0(x,b)$ vendran donats per 
\begin{small}
$$\begin{cases}
P(B>7.5) + P(B<x) - P(x \leq B \leq 7.5) & \text{si }x < 7.5 \\
2\cdot P(B>7.5) + 2\cdot P(B<7.5) - P(B = 7.5) & \text{si }x = 7.5 \\
-1 & \text{si } x > 7.5
\end{cases}$$
\end{small}
on $B$ és la suma total obtinguda per la banca suposant que s'atura a partir de $b$ o més. Per tant, el càlcul de $G_0(x,b)$ depèn exclusivament de les probabilitats $P(B = j)$, $j = 0.5, 1 , 1.5, \cdots , 7.5, > 7.5$. 

Ja que calcular aquestes probabilitats és un procés molt tediós, s'han utilitzat les tècniques de Montecarlo per obtenir-les aproximadament a partir de 200.000 partides simulades per a cada $x$ i per a cada $b$.

Finalment, tenint en compte que si ens passam el guany esperat és sempre $-1$, i que si obtenim 7 punts i mig, el millor que podem fer és aturar-nos, podem obtenir els següents valors finals de $G^*(x,b)$ 
$$\begin{cases}
G^*(x,b)  = -1 & \text{si } x>7.5 \text{ i per a tot } b\\
G^*(7.5,b) = G_0(7.5,b) & \text{per a tot } b
\end{cases}$$
a partir dels quals s'iniciarà el procés d'inducció cap enrere aplicant l'equació recursiva \ref{eqGEstrella}.

\subsection{Quan augmentar l'aposta?}

Ara considerarem la possibilitat d'incrementar l'aposta quan les nostres cartes sumen $x$, recordem que podem incrementar l'aposta sempre i quan demanem una carta més. Així, hem de cercar per a cada $x$ i per a cada $b$, a partir de quina aposta $A$ es compleix (si es compleix) que 
$$A \cdot G_1^*(x,b) := G^*_A(x,b)>G^*(x,b).$$
D'aquesta manera podem aconseguir l'estratègia d'aturada òptima que es pot veure al final de l'article a la taula \ref{EstrategiaOptimaDoblant}, on:

\begin{itemize}
\item $\bigstar$ indica que si demanam una carta més el guany esperat  és positiu i per tant és convenient augmentar l'aposta. 
\item $\triangle$ indica que hem de demanar una altra carta, però sense augmentar l'aposta ja que el guany esperat en aquest cas és negatiu.
\item $\blacksquare$ indica que ens hem d'aturar i, a més, el guany esperat és positiu.
\item $\square$ indica que ens hem d'aturar, a pesar de que el guany esperat sigui negatiu.
\item Els números indiquen el que s'ha d'apostar, com a mínim, per a que sigui més favorable demanar que aturar-se.
\end{itemize}

\section{Guany esperat seguint l'estratègia òptima}

Una vegada determinada l'estratègia òptima, és lògic demanar-se quina és la mitjana de la variable aleatòria guany quan s'aposta un euro i se segueix aquesta estratègia; així, podem calcular l'esperança del guany en funció de l'estratègia de la banca. Si anomenam $E(x,b)$ al guany esperat si les meves cartes sumen $x$ i la banca decideix seguir l'estratègia de no aturar-se fins a que les seves cartes no sumin com a mínim $b$, podem calcular l'esperança $E(b)$ per a cada estratègia que pugui seguir la banca mitjançant
$$E(b) = 0.3 \cdot E(0.5,b) + 0.1 \cdot \sum_{i=1}^7 E(i,b)$$
Realitzant els càlculs, obtenim que l'esperança és negativa per al jugador quan la banca elegeix seguir l'estratègia 4, 4.5, 5 o 5.5, essent la millor (per a la banca) la de no aturar-se amb menys de 5 punts; amb aquesta estratègia, la banca guanyarà, de mitjana, com a mínim un 11\% de les apostes realitzades (recordem que aquesta és la mitjana sempre i quan el jugador jugui de manera òptima, altrament, el percentatge de guanys per a la banca seran majors).

Ara bé, es pot millorar aquesta estratègia? Podem aconseguir que l'esperança del joc sigui sempre positiva per al jugador? La resposta és sí; recordem que si elegim demanar una carta més, podem augmentar l'aposta sense límit, si observam els valors de $G^*(x,b)$ veim que el millor moment per augmentar l'aposta, és quan $x = 0.5$, d'aquesta manera podem calcular els diners que s'han d'apostar quan es rep una carta de valor 0.5 per tal que l'esperança del joc sigui positiva. Aquests diners, diguem-los $d$, vendran determinats per 
$$E(b) = 0.3 \cdot E(0.5,b) \cdot d + 0.1 \cdot \sum_{i=1}^7 E(i,b) \geq 0.$$
Així, realitzant els càlculs per a les estratègies d'esperança negativa obtenim que s'ha d'apostar  com a mínim 2, 4, 7 i 6 per a les estratègies 4, 4.5, 5 i 5.5 respectivament.


D'aquesta manera, podem trobar, per a qualsevol estratègia que pugui seguir la banca, una estratègia òptima que té esperança positiva per al jugador. Per tant, la primera conclusió que es treu és que, si es coneix l'estratègia que seguirà la banca, seguint l'estratègia òptima, el 7 i mig pot convertir-se en un joc favorable per al jugador i, a més a més, apostant suficients diners es pot tenir una esperança de guany tan gran com es vulgui.

\section{Que passa si erram l'estratègia que seguirà la banca?}

En l'estudi realitzat s'ha suposat sempre que es coneix l'estratègia que seguirà la banca, però això no sempre és així. A la taula \ref{avantatge} podem veure que passa si ens equivocam d'estratègia, és a dir, si elegim jugar l'estratègia òptima per a $b$ però en canvi la banca elegeix seguir una altra estratègia $b'$.



Podem veure que no hi ha cap estratègia que ens serveixi per a tenir avantatge enfront totes les altres estratègies; la solució a aquest problema passa, una vegada més, per augmentar l'aposta que realitzam a l'hora d'obtenir un mig (recordem que com més apostam en els mitjos, més guanyam de mitjana). Així, per exemple, si elegim seguir l'estratègia òptima per a $b=1$, apostant sempre com a mínim 11 euros a les cartes de valor 0.5 aconseguim esperança positiva, sigui quina sigui l'estratègia que segueixi la banca.


\section{Enterrar-se}

Recordem que enterrar-se consistia en, si les nostres cartes sumen 7, ensenyar el 7 i demanar una carta que no es destaparà fins al final del joc. 

Hem vist que sota la hipòtesi $H_1$ mai és convenient enterrar-se, això es pot veure calculant l'esperança d'enterrar-se:
\begin{align*}
E &= 2\cdot  A \cdot P(x = 0.5) - A \cdot P(x > 0.5) \\
&= 2\cdot A \cdot 0.3 - A\cdot 0.7 \\
&= -0.1\cdot A
\end{align*}
on $x$ és la carta tapada i $A$ l'aposta realitzada sobre l'enterrament. Això significa que, sota la hipòtesi $H_1$, de mitjana perdrem un 10\% dels diners que apostem.

Ara bé, si obviam la hipòtesi $H_1$ podem trobar condicions sobre $P(x = 0.5)$ perquè l'esperança de l'enterrament sigui positiva, això és
\begin{align*}
E &= 2\cdot  A \cdot P(x = 0.5) - A \cdot P(x \geq 0.5) \\
&= A\cdot (3\cdot P(x = 0.5) - 1) \geq 0
\end{align*}
Per tant ha de passar que 
$$P(x = 0.5) \geq \frac{1}{3}.$$
Ja que $P(x = 0.5) = \frac{12}{40}$, volem que passi a ser de $\frac{12}{36}$, o $\frac{11}{33}$, o $\frac{10}{30}$... És a dir, si comptam els mitjos i les cartes que no són mitjos que han sortit i els anomenam $\alpha$ i $\beta$ respectivament, ens convendrà apostar a l'enterrament quan 
$$\beta \geq 3\alpha + 4$$ 

\section{Una partida simulada}

En Biel, en Jaume, na Catalina i en Marc s'han ajuntat per jugar una partida al joc del 7 i mig. En Marc, que ha escoltat allò de "\textit{la banca sempre guanya}", es presta voluntari per a ser la banca i, seguint les normes del joc, reparteix una carta tapada a cada jugador i una per a ell. Anem a veure com es desenvoluparia el joc si en Biel, en Jaume i na Catalina juguen seguint l'estratègia òptima estudiada anteriorment.

Per a garantir un guany esperat positiu independentment del que decideixi fer la banca, els jugadors haurien d'apostar com a mínim 11 euros a cada carta de valor 0.5; no obstant, cap dels jugadors disposa de tants de diners per invertir en una sola carta, així que hauran d'intentar encertar l'estratègia que seguirà la banca per a poder utilitzar la regla d'aturada òptima de la taula \ref{EstrategiaOptimaDoblant}.

\subsection{Joc de na Catalina}

Na Catalina pensa que en Marc seguirà l'estratègia de demanar fins a no tenir almenys 4.5, és a dir, l'estratègia $b = 4.5$

La carta de na Catalina és un mig, per tant es té que $ x= 0.5$, si miram la taula \ref{EstrategiaOptimaDoblant} trobam una $\bigstar$, lo qual ens indica que hem d'augmentar l'aposta, per això na Catalina decideix augmentar l'aposta a 3 euros i demanar una carta més.

La següent carta que rep és un 1, amb la qual cosa la seva puntuació passa a ser ara de $x= 1.5$, tornant a la taula de l'estratègia òptima ens trobam amb un $\triangle$, per tant hem de demanar una altra carta, però sense augmentar l'aposta.

La següent carta que rep és un 4, ara la puntuació de na Catalina és de $x = 5.5$ i, a la taula \ref{EstrategiaOptimaDoblant}, trobam un $\square$, de manera que ens hem d'aturar de demanar, a pesar de que el guany esperat en aquests moments sigui negatiu.

\subsection{Joc de'n Biel}

En Biel gira amb cura la seva carta i veu és un 7, de manera que considera la possibilitat de realitzar un enterrament augmentant l'aposta a 5 euros. Abans de realitzar-lo, dissimuladament, es fixa amb les cartes que ja han sortit: veu que han sortit 4 cartes, 3 de les quals no són mitjos (0.5,1,4 i 7), això significa que $\alpha = 1$ i $\beta = 3$, ja que no es compleix que $\beta \geq 3\alpha + 4$ i a la taula \ref{EstrategiaOptimaDoblant} trobam un $\blacksquare$ per a totes les estratègies que pugui seguir la banca, decideix aturar-se sabent que la seva esperança és positiva.

\subsection{Joc de'n Jaume}

En Jaume, ja que coneix molt bé en Marc i sap que és un jugador molt agressiu, pensa que demanarà cartes fins a no fer 7 punts i mig.

Observa la seva carta i veu que és un 3; l'estratègia d'aturada òptima per a $b= 7.5$ li indica que ha d'augmentar la seva aposta a 5 euros i demanar una carta més. Això és el que fa i la següent carta que rep és un 1, ara la puntuació de'n Jaume és 4 i, ja que a la taula \ref{EstrategiaOptimaDoblant} trobam un $\blacksquare$, decideix aturar-se.


\subsection{Joc de'n Marc}

Finalment arriba el joc de la banca i, a pesar de que la intenció de'n Marc al principi era anar a fer 7 i mig, intueix que en Jaume pot no tenir una puntuació massa alta, per això decideix jugar una mica més conservador i aturar-se a partir de 5, és a dir, seguir l'estratègia $b=5$. 

Abans de revelar la carta de'n Marc, observem la taula \ref{avantatge} per a determinar quina és l'esperança de cada jugador, tenint en compte que la banca ha elegit l'estratègia $b' = 5$. Na Catalina ha elegit l'estratègia òptima per a $b = 4.5$, a la taula \ref{avantatge} veim un $\blacksquare$, amb la qual cosa, a pesar d'haver-se equivocat a l'hora d'endevinar l'estratègia de la banca, segueix conservant l'avantatge. En canvi en Jaume, que ha elegit l'estratègia $b = 7.5$, té un $\square$, de manera que ha perdut l'avantatge que en un principi l'estratègia d'aturada òptima li havia proporcionat.

Finalment, en Marc destapa la seva carta i resulta ser un 5, de manera que decideix aturar-se. Així, ha de pagar 3 euros a na Catalina i 1 euro a en Biel, ja que aquests han superat la seva puntuació, i ha de cobrar els 5 euros que havia apostat en Jaume, ja que aquest s'havia plantat amb només 4 punts.

\section{Conclusions}

En aquest article s'ha trobat una estratègia d'aturada òptima per al joc del 7 i mig amb esperança positiva per al jugador. No obstant, aquest guany esperat positiu és a costa d'apostar una quantitat bastant elevada de diners, i això, degut a que la variància del joc és gran, pot acabar en pèrdues elevades a curt plaç, que ens poden portar a la bancarrota si no disposam d'un capital inicial suficientment gran com per a poder afrontar-les. 

Tenir la capacitat de preveure l'estratègia que seguirà la banca ens pot servir per a reduir la quantia de les apostes, no obstant, com hem vist a la partida simulada anterior, errar en les prediccions ens pot conduir a prendre una decisió equivocada. Aquest desavantatge és veu empitjorat pel fet de que la banca sempre fa el seu joc al final, després de veure el joc dels jugadors. Per això, una banca intel·ligent pot adaptar l'estratègia a seguir en funció del joc i de les cartes que tenen destapades els jugadors, essent aquest un dels avantatges més importants que posseeix la banca (cas de'n Marc i en Jaume per exemple) i que fa valer el dit popular 
\begin{center}
"\textit{La banca sempre guanya}"
\end{center}

\section*{Agraïments} 
Aquest article forma part del treball realitzat a l'assignatura optativa de la Llicenciatura de Matemàtiques "Treball acadèmicament dirigit", dirigit pel Dr. Jaume Suñer Llabrés.



\iffalse
\bibliography{biblioPFC}
\bibliographystyle{plain}
\fi

\section*{Bibliografia}
[1] Juan Tejada Cazorla. Javier Yañez Gestoso. Departament d'Estadística i Investigació Operativa de la Facultat de Ciències Matemàtiques de l'Universitat Completense de Madrid. \textit{Estudio de la estrategia óptima para el black-jack.} Estadística Española. num 107, 1985, págs 95 a 110.
\end{thebibliography}

\begin{table*}[htb]

\begin{center}
\begin{tabular}{rrrrrrrrrrrrrrrr}
  \hline
$x/b$ & 0.5 & 1 & 1.5 & 2 & 2.5 & 3 & 3.5 & 4 & 4.5 & 5 & 5.5 & 6 & 6.5 & 7 & 7.5 \\ 
  \hline
0.5 &$\bigstar$  & $\bigstar$  & $\bigstar$  &$\bigstar$  & $\bigstar$  & $\bigstar$  & $\bigstar$  & $\bigstar$  & $\bigstar$  &$\bigstar$  & $\bigstar$  & $\bigstar$  &$\bigstar$  & $\bigstar$  & $\bigstar$  \\ 
  1 & $\bigstar$  &$\bigstar$ & $\bigstar$  & $\triangle$ & $\triangle$ & $\triangle$ & $\triangle$ & $\triangle$ & $\triangle$ & $\triangle$ & $\triangle$ & $\triangle$ & 3 & 2 & 2 \\ 
  1.5 & $\bigstar$  & $\bigstar$  & $\bigstar$  &$\bigstar$ & $\bigstar$  & $\bigstar$ &$\bigstar$ &$\bigstar$  & $\triangle$ & $\triangle$ & $\triangle$ & $\bigstar$  & $\bigstar$  & $\bigstar$  & 2 \\ 
  2 & $\bigstar$ & $\triangle$ &$\triangle$ & $\triangle$ & $\triangle$ & $\triangle$ & $\triangle$ & $\triangle$ & $\triangle$ & $\triangle$ & $\square$ & $\square$ &  $\blacksquare$ & $\blacksquare$ & 2 \\ 
  2.5 &$\bigstar$ & $\bigstar$ & $\bigstar$ & $\triangle$ &$\triangle$ & $\triangle$ & $\triangle$ & $\triangle$ & $\triangle$ & $\triangle$ &$\triangle$ &$\triangle$ & 2 & 2 & 2 \\ 
  3 & $\square$ & $\triangle$ &$\triangle$ & $\triangle$ & $\triangle$ & $\triangle$ & $\triangle$ & $\triangle$ & $\triangle$ & $\triangle$ & $\square$ & $\square$ & $\blacksquare$ & $\blacksquare$ & 5 \\ 
  3.5 & 2 & $\bigstar$ & $\triangle$ &$\triangle$ & $\triangle$ & $\triangle$ & $\triangle$& $\triangle$ & $\triangle$ & $\triangle$ & $\triangle$ & $\square$ & $\square$ & 6 & 3 \\ 
  4 & $\blacksquare$ & $\square$ & $\square$ & $\square$ & $\triangle$ & $\triangle$ & $\triangle$ & $\triangle$ & $\triangle$ & $\square$ & $\square$ & $\square$ & $\blacksquare$& $\blacksquare$ &$\blacksquare$ \\ 
  4.5 &7 & $\blacksquare$ & $\blacksquare$ & $\square$ &$\square$ & $\triangle$ & $\triangle$ & $\triangle$ & $\triangle$ & $\triangle$ & $\square$ & $\square$ & $\blacksquare$ & $\blacksquare$ & 9 \\ 
  5 & $\blacksquare$ & $\blacksquare$ & $\blacksquare$ & $\blacksquare$ & $\square$ &$\square$ & $\square$ & $\square$ & $\square$ &$\triangle$ & $\square$& $\square$ & $\blacksquare$ &$\blacksquare$& $\blacksquare$ \\ 
  5.5 &  $\blacksquare$ & $\blacksquare$ & $\blacksquare$ & $\blacksquare$ & $\blacksquare$ & $\blacksquare$ & $\blacksquare$ & $\blacksquare$ & $\square$ & $\square$ & $\square$ & $\square$ & $\blacksquare$ & $\blacksquare$ & $\blacksquare$  \\ 
  6 & $\blacksquare$ & $\blacksquare$ & $\blacksquare$ & $\blacksquare$ &$\blacksquare$ & $\blacksquare$ & $\blacksquare$ & $\blacksquare$ & $\blacksquare$ & $\blacksquare$ & $\square$ & $\square$ & $\blacksquare$ & $\blacksquare$ & $\blacksquare$ \\ 
  6.5& $\blacksquare$ & $\blacksquare$ & $\blacksquare$ & $\blacksquare$ & $\blacksquare$ & $\blacksquare$& $\blacksquare$& $\blacksquare$&$\blacksquare$ & $\blacksquare$ & $\blacksquare$ & $\blacksquare$ & $\blacksquare$& $\blacksquare$ & $\blacksquare$\\ 
  7 & $\blacksquare$ & $\blacksquare$ & $\blacksquare$ & $\blacksquare$ & $\blacksquare$ & $\blacksquare$& $\blacksquare$& $\blacksquare$&$\blacksquare$ & $\blacksquare$ & $\blacksquare$ & $\blacksquare$ & $\blacksquare$& $\blacksquare$ & $\blacksquare$\\ 
  7.5 & $\blacksquare$ & $\blacksquare$ & $\blacksquare$ & $\blacksquare$ & $\blacksquare$ & $\blacksquare$& $\blacksquare$& $\blacksquare$&$\blacksquare$ & $\blacksquare$ & $\blacksquare$ & $\blacksquare$ & $\blacksquare$& $\blacksquare$ & $\blacksquare$\\ 
   \hline
\end{tabular}
\end{center}
\caption{\textit{Estratègia d'aturada òptima}}
\label{EstrategiaOptimaDoblant}

\begin{center}
\begin{tabular}{rlllllllllllllll}
  \hline
$b/b'$ & 0.5 & 1 & 1.5 & 2 & 2.5 & 3 & 3.5 & 4 & 4.5 & 5 & 5.5 & 6 & 6.5 & 7 & 7.5 \\ 
  \hline
0.5 & $\blacksquare$ &$\blacksquare$ & $\blacksquare$ & $\blacksquare$ & $\square$ & $\square$ & $\square$ & $\square$ & $\square$ & $\square$ & $\square$ & $\square$ &$\blacksquare$& $\blacksquare$ & $\blacksquare$ \\ 
  1 & $\blacksquare$ &$\blacksquare$ & $\blacksquare$ & $\blacksquare$ & $\blacksquare$ &$\blacksquare$& $\square$ & $\square$ & $\square$ & $\square$ & $\square$ & $\square$ & $\blacksquare$ & $\blacksquare$ & $\blacksquare$ \\ 
  1.5 & $\blacksquare$ & $\blacksquare$ & $\blacksquare$ & $\blacksquare$ & $\blacksquare$ & $\blacksquare$ & $\square$ & $\square$ & $\square$ & $\square$ & $\square$ & $\square$ & $\blacksquare$ & $\blacksquare$ & $\blacksquare$ \\ 
  2 & $\blacksquare$ &$\blacksquare$ & $\blacksquare$& $\blacksquare$ & $\blacksquare$ & $\blacksquare$ & $\square$ & $\square$ & $\square$ & $\square$ & $\square$ &$\square$ & $\blacksquare$ & $\blacksquare$ & $\blacksquare$ \\ 
  2.5 & $\blacksquare$ & $\blacksquare$ & $\blacksquare$ &$\blacksquare$ & $\blacksquare$ & $\blacksquare$& $\blacksquare$ & $\square$ & $\square$ & $\square$ & $\square$ & $\square$ & $\blacksquare$ & $\blacksquare$ &$\blacksquare$ \\ 
  3 &$\blacksquare$ & $\blacksquare$ & $\blacksquare$ & $\blacksquare$ & $\blacksquare$ & $\blacksquare$&$\blacksquare$ &$\square$ & $\square$ & $\square$ & $\square$ & $\square$ & $\square$ & $\blacksquare$ & $\blacksquare$ \\ 
  3.5 & $\blacksquare$ & $\blacksquare$ & $\blacksquare$ & $\blacksquare$ &$\blacksquare$ & $\blacksquare$ & $\blacksquare$ & $\square$ & $\square$ & $\square$ & $\square$ & $\square$ & $\square$ & $\blacksquare$ & $\blacksquare$ \\ 
  4 & $\blacksquare$ &$\blacksquare$ & $\blacksquare$ & $\blacksquare$ & $\blacksquare$ & $\blacksquare$ & $\blacksquare$ &$\blacksquare$ &$\square$ & $\square$& $\square$ & $\square$ & $\blacksquare$ & $\blacksquare$ & $\blacksquare$ \\ 
  4.5 & $\blacksquare$ & $\blacksquare$ & $\blacksquare$ & $\blacksquare$ & $\blacksquare$ & $\blacksquare$ & $\blacksquare$ & $\blacksquare$ & $\blacksquare$ & $\square$ &$\square$ & $\square$ & $\blacksquare$ &$\blacksquare$ & $\blacksquare$ \\ 
  5 & $\blacksquare$ & $\blacksquare$ & $\blacksquare$& $\blacksquare$ & $\blacksquare$& $\blacksquare$ & $\blacksquare$ &$\blacksquare$ & $\blacksquare$ & $\blacksquare$ & $\square$ & $\square$ & $\blacksquare$ & $\blacksquare$ & $\blacksquare$ \\ 
  5.5 & $\blacksquare$ & $\blacksquare$ & $\blacksquare$ & $\blacksquare$ & $\blacksquare$ & $\blacksquare$ & $\square$ & $\square$& $\square$& $\square$& $\blacksquare$ & $\blacksquare$ & $\blacksquare$ & $\blacksquare$ & $\blacksquare$ \\ 
  6 & $\blacksquare$ & $\blacksquare$ & $\blacksquare$ &$\square$ &$\square$ &$\square$ & $\square$& $\square$ & $\square$ & $\square$ & $\square$ & $\blacksquare$ & $\blacksquare$ &$\blacksquare$& $\blacksquare$ \\ 
  6.5 & $\blacksquare$ & $\blacksquare$ &$\square$ & $\square$ & $\square$ & $\square$ & $\square$& $\square$ & $\square$ & $\square$ & $\square$ &$\square$ &$\blacksquare$ & $\blacksquare$ & $\blacksquare$ \\ 
  7 & $\blacksquare$ & $\blacksquare$ & $\square$ & $\square$ &$\square$ &$\square$ & $\square$ & $\square$& $\square$ & $\square$& $\square$ & $\square$& $\blacksquare$ & $\blacksquare$ & $\blacksquare$ \\ 
  7.5 &$\blacksquare$ & $\square$ & $\square$ &$\square$ &$\square$ & $\square$ & $\square$& $\square$&$\square$ & $\square$ &$\square$ & $\square$ & $\square$ & $\square$ & $\blacksquare$ \\ 
   \hline
\end{tabular}
\end{center}
\caption{\textit{Seguim l'estratègia òptima per a $b$ però en canvi la banca segueix $b'$\\
$\blacksquare$: Avantatge per al jugador $\square$: avantatge per a la banca}}
\label{avantatge}
\end{table*}


\end{document} 