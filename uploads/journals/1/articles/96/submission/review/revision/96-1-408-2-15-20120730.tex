%Ejemplo de Paper para JITEL 2011


\newcommand{\CLASSINPUTinnersidemargin}{18mm}
\newcommand{\CLASSINPUToutersidemargin}{12mm}
\newcommand{\CLASSINPUTtoptextmargin}{20mm}
\newcommand{\CLASSINPUTbottomtextmargin}{25mm}
\documentclass[10pt,journal,a4paper]{IEEEtran}
%%%%%%%%%%%%%%%%%%%%%%%%%%%%%%%%%%%%%%%%%%%%%%%%%%%%%%%%%%%%%%%%%%%%%%%%%%%%%%%%%%%%%%%%%%%%%%%%%%%%%%%%%%%%%%%%%%%%%%%%%%%%%%%%%%%%%%%%%%%%%%%%%%%%%%%%%%%%%%%%%%%%%%%%%%%%%%%%%%%%%%%%%%%%%%%%%%%%%%%%%%%%%%%%%%%%%%%%%%%%%%%%%%%%%%%%%%%%%%%%%%%%%%%%%%%%

% para que salgan las tildes y dem�s caracteres en castellano...
%\usepackage[latin1]{inputenc}
\usepackage[T1]{fontenc}
\usepackage[catalan]{babel}

\usepackage{amssymb}

%o bien
\usepackage[utf8]{inputenc}

% Fuente Times...
\usepackage{times}


% figuras en formato .png, .ps, pdf � eps
\usepackage{graphicx}
\DeclareGraphicsExtensions{.png,.eps,.ps,.pdf}

\usepackage[colorlinks]{hyperref}

% formato y tipograf�a de URL, direcciones de correo...
\usepackage{url}

\begin{document}

% Nombres y direcciones de los autores
\title{Ràdio Cognitiva}

\author{
\IEEEauthorblockN{Bartolomé Solera Ripoll, Pedro Barea Jaume, Javier González Maimó}\\
\IEEEauthorblockA{
\emph{3r d'Enginyeria Tècnica en Telecomunicacions, especialitat en Telemàtica} \\
\texttt{tolito\_nevat@hotmail.com, pedrobareajaume@gmail.com,javig181@gmail.com}}
}

\maketitle

\begin{abstract}
Avui en dia, podem dir que hi ha una saturació de l’espectre radioelèctric a
tot el món, on cada país té assignades bandes de freqüències, les quals compleixen una funció específica per a cada un d’ells. Per això, en vista que estan creixent
el nombre de serveis que utilitzen l’espectre (Wi-Fi, ràdio comercial, FM,
AM, radioaficionats, TDT, transmissions reservades al govern per defensa \ldots )
s’ha d’utilitzar algun mètode per a poder aprofitar-lo al màxim, perquè pugui ser
utilitzat per a molts serveis i tots ells amb una QoS acceptable: la ràdio cognitiva.
\end{abstract}

%\begin{keywords}
%
% Palabras 
%
%\end{keywords}

\section{Introducció}
Un dels problemes més greus de l'espectre radioelèctric, representat a la figura \ref{logotipo}, és el desaprofitament dels espais que queden buits entre serveis d'usuaris que tenen una llicència per transmetre a l'espectre (operadors de telefonia, de televisió, ràdios comercials \ldots). Per això, s'han ideat sistemes anomenats de ràdio cognitiva. 
\begin{figure}[tpbh]
\centering
\includegraphics[width=8cm]{spectrum-radiation}
\caption{Utilització de l'espectre radioelèctric} %imatge extreta de: %http://cqham.blogspot.com/2006_08_01_archive.html
\label{logotipo}
\end{figure}
Els sistemes de ràdio cognitiva estan basats en \textbf{ràdio definida per software} (SDR, de l'anglès Software Defined Radio), que proporciona eines en temps real per poder fer front a les necessitats en cada moment dels usuaris.
El SDR és un conjunt de software que realitza les mateixes funcions que el hardware d'un aparell de ràdio.


Per a poder dur a terme la idea principal de ràdio cognitiva s'han de complir alguns requisits per a poder mantenir la qualitat de la transmissió tant dels usuaris amb llicència (Usuaris primaris) com dels usuaris sense llicència (Usuaris secundaris):
\begin{itemize}
\item \emph{Detecció y anàlisi espectral:} el sistema ha de ser capaç de detectar les bandes de l'espectre lliures per a poder assignar-les als usuaris.
\item \emph{Mobilitat espectral:} el sistema ha de ser capaç d'adaptar-se a les condicions en temps real per a poder adaptar canvis de bandes freqüencials i detecció de nous espais lliures de forma transparent per a l'usuari.
\item \emph{Compartir els recursos espectrals:} el sistema, a més, ha de ser capaç de mantenir la qualitat de transmissió de tot l'espectre evitant les interferències entre serveis que tenen llicència per transmetre a una banda de freqüències i altres serveis que no en tenen i transmeten a bandes lliures. 
\end{itemize}
%%%%%%%%%%%%%%%%%%%%%%%%%%%%%%%%%
\subsection{El concepte de Ràdio Cognitiva}
Va ser introduït per Joseph Mitola III en un article escrit per ell i per 
Gerald Q. Maguire a l'any 1999, però realment on es va definir la idea va ser a la tesi doctoral Mitola anomenada \emph{“Cognitive Radio: An Integrated
Agent Arquitecture for Software Defined Radio”} de l'any 2000, on defineix la ràdio cognitiva com
\emph{la identificació del punt  on les PDA's (Personal Digital Assistants) sense fils i les xarxes relacionades, de forma computacional, son suficientment intel·ligents sobre els recursos radioelèctrics, i les comunicacions entre ordinadors per detectar les necessitats de comunicació dels usuaris en funció de l'ús i
proveïr els recursos ràdio i els serveis sense fils més apropiats per les necessitats de cada un \cite{III_2000}.}
%biblio d'això: http://www.mendeley.com/research/cognitive-radio-integrated-agent-architecture-software-defined-radio-dissertation/

Es pot dir que abans de Mitola ja hi havia unes definicions de sistemes que podien decidir sobre el seu funcionament \cite{fetweiss} i, de fet, a la ràdio cognitiva se la considerava com una extensió de Software Defined Radio.

Però la FCC (Federal Communications Commission) nordamericana li va donar una altra definició: \emph{la ràdio cognitiva és un sistema ràdio capaç de canviar els seus paràmetres de transmissió basant-se en la interacció amb l'entorn en què opera.}

Mitola també introdueix el concepte de \emph{Spectrum Polling}, que és un repartiment d'espectre entre usuaris primaris i usuaris secundaris.
%biblio d'això: http://www.mendeley.com/research/cognitive-radio-integrated-agent-architecture-software-defined-radio-dissertation/

%%%%%%%%%%%%%%%%%%%%%%%%%%%%%%%%%%%
\subsection{Estandardització}
Vista la complexitat tant tècnica com econòmica per a poder implementar sistemes de ràdio cognitiva, cal una estandardització d'aquests.
Per poder solventar problemes de compatibilitat, s'ha creat  l' \textbf{IEEE Standards Coordinating Commitee (SCC) 41} que és l'encarregat d'estandarditzar l'\textbf{IEEE 1900} \cite{ieee_1900}.

Aquest estàndard té els següents components: 

\begin{itemize}
\item \textbf{IEEE 1900.1:} estudia la terminologia  i conceptes relacionats amb Next Generation Radio and Spectrum Management com la gestió del espectre, SDR \ldots
\item \textbf{IEEE 1900.2:} estudia i analitza l'interferència i coexistència dels canals.
\item \textbf{IEEE 1900.3:} s'avaluen diferents conceptes  sobre SDR.
\item \textbf{IEEE 1900.4:} estudia com es podran connectar els dispositius dels usuaris a entorn sense fils heterogenis.
\item \textbf{IEEE 1900.5:} estudia l'arquitectura i el llenguatge utilitzat  tant a la gestió de ràdio cognitiva, com a l'accés a l'espectre 
\item \textbf{IEEE 1900.6:} estudia estructures de dades i interfícies DSA i d'altres sistemes de comunicacions ràdio.
\item \textbf{IEEE 1900.7:} estudia la interfície ràdio per a nous sistemes mòbils i altres sistemes d'accés.
\end{itemize}
Cal dir que l'IEEE 1900 sovint és anomenat DySPAN SC.
%%%%%%%%%%%%%%%%%%%%%%%%%%%%%%%%%%%
\section{Sistemes de Ràdio Cognitiva}
\subsection{Arquitectura}
L'arquitectura de la ràdio cognitiva és totalment independent de la interfície ràdio a la qual s'aplica.
Un sistema de ràdio cognitiva és un conjunt d'algorismes implementats mitjançant software , els quals es compaginen amb una interfície ràdio.
El motor cognitiu ha de ser capaç de detectar tant els dispositius com les bandes lliures de l'espectre, aprendre l'estructura d'aquests i prendre decisions que compleixin una QoS determinada.
El motor cognitiu  és una interfície software, però s'ha d'implementar sobre el hardware de la interfície ràdio, per a poder realitzar l'operació de la millor forma possible.
%%%%%%%%%%%%%
\begin{figure}
  \centering
 \includegraphics[width=0.5\textwidth]{proc_rad_cog}
  \caption{Funcionament d'un sistema amb Ràdio Cognitiva \cite{renter_2011}}
  \label{fig:proc_rad}
\end{figure}

Tal com il·lustra la figura \ref{fig:proc_rad}, un sistema de Ràdio Cognitiva funciona de la següent manera:
\begin{itemize}
\item \textbf{Transmissor/Receptor:} transmet i rep informació, tant dels usuaris com de l'espectre.
\item \textbf{Analitzador d'espectres:} amb els senyals capturats analitza l'espectre per a trobar els espais lliures  i per a no interferir amb els usuaris primaris.
\item \textbf{Mòdul d'aprenentatge:} s'utilitza per a cercar informació de l'analitzador d'espectres i recopilar-la. El mòdul d'aprenentage ha de saber interpretar la informació rebuda sobre la utilització de l'espectre en cada moment.
\item \textbf{Mòdul de decisió:} s'encarrega de la decisió que es pren per a poder entrar a l'espectre. Empra la informació del mòdul d'aprenentage i s'utilitzen criteris relacionats amb l'entorn per a poder prendre una decisió. 
\end{itemize} 
%%%%%%%%%%%%%%%%%%%%%%%%%%%%%%%%%%%%
\subsection{Accés al medi: Dynamic Spectrum Access}
El Dynamic Spectrum Access (DSA) defineix la manera d'accedir a l'espectre radioelèctric depenent de les condicions que es donin en cada moment . En general, es pot dir que l'accés es duu a terme tal com mostra la figura \ref{dsa}, malgrat això, hi ha dos tipus de models DSA. 
El primer model diferencia el tipus d'accés dels usuaris a l'espectre \cite{garwhal_2011}:
\begin{enumerate}
\item \textbf{DSA d'ús exclusiu:} un usuari primari dóna permís a un usuari secundari perquè pugui transmetre exclusivament a l'espectre.
\item \textbf{DSA d'ús compartit:} els usuaris secundaris accedeixen a l'espectre quan hi ha espectre lliure, però sempre, sense interferir en els altres usuaris primaris.
Dins dels d'ús compartit existeixen dues alternatives:
\begin{enumerate}
\item \textbf{Overlay:} l'usuari secundari té accés a l'espectre d'un usuari primari que en aquell moment no transmet, fet que implica una monitorització de l'usuari secundari amb el primari.
\item \textbf{Underlay:} l'usuari secundari transmet al mateix moment que l'usuari primari i, per tant, tendrà una potència limitada per tal de no interferir en la transmissió d'aquest darrer. 
\end{enumerate}
\item \textbf{Models comuns:} s'accedeix a l'espectre de forma lliure sense cap limitació. Es suposa que a l'espectre no hi ha usuaris primaris que ocupin bandes o donin permís per accedir-hi a elles.
\end{enumerate}
\begin{figure}
\centering
\includegraphics[width=9cm]{dsa_cat}
\caption{Accés al medi (DSA)}
\label{dsa}
\end{figure}
El segon model, determina la gestió per a accedir a l'espectre \cite{garwhal_2011}:
\begin{enumerate}
\item \textbf{Implementació centralitzada:} hi ha un control central per a gestionar els accessos a l'espectre depenent de les necessitats dels usuaris secundaris que vulguin accedir-hi. A més, es recopila informació sobre la utilització d'aquest.
\item \textbf{Implementació distribuïda:} On els usuaris secundaris accedeixen  a l'espectre de forma autònoma i segons la informació que tenen sobre ell. (No és tan eficient com la primera).
\end{enumerate}
%%%%%%%%%%%%%%%%%%%%%%%%%%%%%%%%%%%%%
\section{Projectes de Ràdio Cognitiva}
\subsection{El projecte de Timo Weiss i Friedrich Jondral}
Arquitectura basada en ràdio cognitiva desenvolupada per T. Weiss y F. Jondral (Universitat de Karlsruhe, Alemanya), que destaca per l'utilització de OFDM.
Suposa que els usuaris primaris sempre ocupen la mateixa banda freqüencial i no suporten canvis al respecte.  Per part dels usuaris secundaris, es necessita, per al funcionament del sistema, una estació base i diversos terminals mòbils, tots, amb ràdio cognitiva \cite{Weiss_Jondral_2004}.
L'estació base recopila informació sobre els usuaris primaris que hi ha a l'espectre i envia missatges de broadcast als usuaris secundaris (en la fase d'aprenentatge).
Llavors, els dispositius es monitoritzen i tornen la informació a l'estació base que la recopila per a donar l'accés a aquests.
Però aquesta arquitectura no solucionava el problema de la interferència que hi havia entre usuaris secundaris i també entre aquests i usuaris primaris. Per a evitar això es va decidir deixar unes "bandes de guarda" o espais entre usuaris primaris i secundaris i entre dos d'aquests darrers, així, mai s'assignarien usuaris a freqüències portadores adjacents. \cite{Weiss_Jondral_2004}
%%%%%%%%%%%%%%%%%%%%%%%%%%%%%%
\subsection{El projecte XG DARPA}
Simplement generalitza a Weiss i Jondral. DARPA (Defense Advanced
Research Projects Agency) va iniciar el programa neXt Generation (XG), en el qual definien una arquitectura que utilitzava SDR, però controlant-la mitjançant diferents polítiques \cite{Mchenry_2007}:
\begin{enumerate}
\item Monitorització de l'entorn.
\item Restringir en relació als canals disponibles i no disponibles, depenent de les característiques dels canals.
\item Com accedir als canals.
\end{enumerate}
%%%%%%%%%%%%%%%%%%%%%%%%%%%%
\subsection{El projecte CORVUS}
Va ser iniciat per les Universitats de Berkeley (USA) i Berlín (Alemanya).
Aquest projecte de ràdio cognitiva es basa en l'abundància de canals disponibles per usuaris secundaris i les tècniques de ràdio cognitiva que utilitzen per evitar les interferències amb els usuaris primaris.

En aquesta arquitectura, els usuaris primaris no emeten senyalització per identificar-se a l'espectre. A més, els usuaris secundaris sense capacitat de ràdio cognitiva es tracten com a renou.
El projecte CORVUS defineix un mètode per accedir a les bandes ocupades per un usuari primari sempre i quan no estiguin ocupades per ells. En aquest cas, els usuaris secundaris disposaran d'un temps en el qual provocaran interferència a l'usuari primari que vol transmetre per la seva banda. Una vegada superat aquest temps els usuaris secundaris han de ser capaços d'aturar la transmissió en aquesta banda  i cercar-ne una altra de lliure.
\begin{figure}
%tbph
\centering
\includegraphics[width=9cm]{corvus}
\caption{Repartiment de l'espectre a CORVUS}
\label{ref:corvus}
\end{figure}

En cas de no haver deixat la banda freqüencial, l'usuari primari començarà a transmetre a la mateixa banda sense donar cap avís a l'usuari secundari \cite{corvus}. 

L'arquitectura també preveu un sistema per a què els usuaris secundaris detectin als usuaris primaris, això s'anomena \textbf{Primary User Footprints (PUF)}, que inclou informació  sobre els usuaris primaris. Això es fa abans què un usuari secundari s'introdueixi dins l'espectre, ja que s'ha de monitoritzar per a poder afegir-se correctament. Aquest procés s'haurà de fer periòdicament perquè si és a una banda que està ocupada per un usuari primari, l'haurà de deixar com més aviat, millor.
Els usuaris secundaris, a CORVUS, formen els \textbf{Secondary User Groups, SUG} que han d'estar coordinats per poder realitzar les comunicacions i poder comunicar-se entre ells.

Per part dels usuaris secundaris, hi ha dos tipus de tràfic definit:
\begin{enumerate}
\item \emph{Tràfic web:} cerquen l'accés a Internet i, per tant, una connexió a una estació base o un punt d'accés
\item \emph{Xarxa ad-hoc:} accepta tot tipus de tràfic sense cap tipus d'infraestructura.
\end{enumerate}
A CORVUS es pot donar el fet que dos SUG utilitzin el mateix rang de freqüències i, per tant, s'hagin d'anar administrant l'espai.

Es defineixen uns canals de control per l'intercanvi d'informació entre usuaris secundaris:
\begin{itemize}
\item Canal de Control Universal (Universal Control Channel, UCC)
\item Canals de Control de Grups (Group Control Channels, GCC)
\end{itemize}
El més important és l'UCC, el qual defineix els grups d'usuaris secundaris que hi ha i permet la incorporació de cada usuari. Cada grup de secundaris, té un GCC per a comunicar-se entre els membres del grup.
Aquest esquema de canals es pot apreciar a la figura \ref{ref:corvus} \cite{corvus}.


%\begin{figure}[tbph]
%\centering
%\includegraphics[width=9cm]{dsa_cat}
%\caption{Accés al medi (DSA)}
%\label{logotipo}
%\end{figure}
\section{Perspectives i aplicacions}
Ja s'han realitzat estudis sobre l'aplicació de la ràdio cognitiva i sobre l'optimització (FCC), sobretot, basat en l'aprofitament de l'espectre en zones on està infrautilitzat (per exemple, zones rurals, on s'utilitza un 15 \% de l'espectre). 
Per altra part, molts països han reconegut que el seu espectre està molt saturat, en conseqüència, estan estudiant nous sistemes per optimitzar-lo.

\subsection*{IEEE 802.22} \label{sub:ieee}
L'IEEE 802.22 és un estàndard que defineix els WRAN (Wireless Regional Area Network), que dóna serveis de banda ampla a zones amb poca densitat de població (zones rurals, zones inhòspites \ldots) utilitzant els espais lliures de l'espectre de freqüències utilitzades per els serveis de televisió.

Aquest estàndard utilitza tècniques de ràdio cognitiva per a permetre l'accés dels usuaris a l'espectre. Aquests usuaris accedirien a l'espectre assignat a la TV (és un servei amb llicència o primari), com es pot veure a la figura \ref{wran}. 
Evidentment, el servei que defineix l'IEEE 802.22 no ha d'interferir en les emissions de televisió .


És una tècnica que empra ràdio cognitiva com a sistema per accedir al medi. A més, és molt recent, fet que fa pensar que la ràdio cognitiva encara no és un sistema molt utilitzat, malgrat hi hagi distints projectes que s'estan desenvolupant.



La manera de conèixer l'espectre que es proposa des de diversos grups que desenvolupen l'estàndard es basa en
\begin{figure}
  \centering
 \includegraphics[width=0.5\textwidth]{wran}
  \caption{Estructura xarxa WRAN}
  \label{wran}
\end{figure}
les estacions base utilitzades per les emissions de TV, les quals, hauran d'incorporar un GPS que informi de la localització a un servidor centralitzat. L'estació base respondrà amb els canals de TV que ocupen la banda (usuaris primaris, en aquest cas) i les seves bandes de guarda \cite{wran_wiki}. 

Aquest estàndard especifica que el sistema estarà format pel conjunt d'estacions base, repetidors de TV i equipaments tècnics i l'equip local de client (CPE, de l'anglès Customer Premises Equipment) o usuaris secundaris, als quals, se'ls diferencia segons el tipus de dispositiu: els de dispositiu mòbil i els dispositiu fix, que duran incorporats funcions GPS per a poder rebre informació i conèixer l'ocupació de l'espectre que correspon a la seva zona (altres transmissors, usuaris primaris\dots). 
Per altra part, els grups que desenvolupen l'estàndard estan discutint una sèrie de característiques importants del protocol com, la centralització del servei de recopilació d'informació. 

A causa de la publicació recent  de l'estàndard (juliol de 2011), podem dir que encara no està molt desenvolupat \cite{wran_wiki}.

% Ejemplo de Tabla:
%
%\begin{table}
%\renewcommand{\arraystretch}{1.3}
%\caption{An Example of a Table}
%\label{table_example}
%\begin{center}
%\begin{tabular}{|c||c|}
%\hline
%One & Two\\
%\hline
%Three & Four\\
%\hline
%\end{tabular}
%\end{center}
%\end{table}

\section{Conclusions}
L'espectre radioelèctric  està molt saturat degut a la quantitat de tecnologies que l’utilitzen. Aquest fet ha motivat als governs i als investigadors a cercar un mètode de millora de l'aprofitament de l’espectre, la ràdio cognitiva, que és una possible solució al greu problema que es planteja.
Es tracta d’una tecnologia molt recent; recordem que Joseph Mitola introdueix el concepte l’any 1999, l’estàndard IEEE 1900 que defineix conceptes com l’SDR i el DSA es comença a plantejar  l’any 2005, i les aplicacions que poden incorporar ràdio cognitiva com, per exemple, es diu a la secció \ref{sub:ieee} son molt recents. Degut a n’aquest fet, encara no s’han implementat molts sistemes que l’utilitzin, malgrat com hem pogut veure, està molt definit.

Cal incidir en la importància d’aquesta tecnologia en el futur, ja que, cada vegada; l’espectre radioelèctric, a pràcticament, tot el món, està més saturat (serveis de TV, ràdio, WI-FI, bandes reservades \ldots).
Per tant, és una tecnologia que s’hauria d’introduïr el més prest possible. Això implica un augment de les investigacions per a aconseguir la implementació de la ràdio cognitiva arreu del món.
\\
\\
\\
\\
\newline
Aquest article pertany a l'assignatura Xarxes de Comunicacions Mòbils. Impartida per Jaume Ramis Bibiloni.
%Bibliograf�a
\bibliographystyle{IEEEtran}
\bibliography{biblio_mobils}
%keepaspectratio
\begin{IEEEbiography}[{\includegraphics[width=0.8in,height=0.9in,clip,keepaspectratio]{./xavi.png}}]{Javier González Maimó}
Estudiant d'Enginyeria Tècnica en Telecomunicacions, especialitat Telemàtica a la UIB. Actualment cursant el PFC.
\end{IEEEbiography}
\begin{IEEEbiography}[{\includegraphics[width=0.8in,height=0.9in,clip,keepaspectratio]{./ToloSolera.jpg}}]{Tolo Solera Ripoll}
Estudiant d'Enginyeria Tècnica en Telecomunicacions , especialitat Telemàtica i Enginyeria Tècnica Industrial,especialitat en Electrònica Industrial  a la UIB. Actualment cursant el PFC.
\end{IEEEbiography}

\begin{IEEEbiography}[{\includegraphics[width=0.8in,height=1.1in,clip,keepaspectratio]{./foto_carnet.png}}]{Pedro Barea Jaume }
Estudiant d'Enginyeria Tècnica en Telecomunicacions , especialitat Telemàtica. Actualment cursant el PFC.
\end{IEEEbiography}

\end{document}
